\documentclass[review]{elsarticle}

\usepackage{lineno,hyperref}
\modulolinenumbers[1]

\journal{Comparative Biochemistry and Physiology}
\journal{Physiological Entomology}
\journal{Environmental Entomology}
\journal{Chemoecology}

%%%%%%%%%%%%%%%%%%%%%%%
%% Elsevier bibliography styles
%%%%%%%%%%%%%%%%%%%%%%%
%% To change the style, put a % in front of the second line of the current style and
%% remove the % from the second line of the style you would like to use.
%%%%%%%%%%%%%%%%%%%%%%%

%% Numbered
%\bibliographystyle{model1-num-names}

%% Numbered without titles
%\bibliographystyle{model1a-num-names}

%% Harvard
%\bibliographystyle{model2-names.bst}\biboptions{authoryear}

%% Vancouver numbered
%\usepackage{numcompress}\bibliographystyle{model3-num-names}

%% Vancouver name/year
%\usepackage{numcompress}\bibliographystyle{model4-names}\biboptions{authoryear}

%% APA style
%\bibliographystyle{model5-names}\biboptions{authoryear}

%% AMA style
%\usepackage{numcompress}\bibliographystyle{model6-num-names}

%% `Elsevier LaTeX' style
\bibliographystyle{elsarticle-num}
%%%%%%%%%%%%%%%%%%%%%%%

\begin{document}

\begin{frontmatter}

\title{Diapause genotype alters triglyceride accumulation during diapause in larvae of \textit{Ostrinia nubilalis} (Lepidoptera: Crambidae)}
\tnotetext[mytitlenote]{Fully documented templates are available in the elsarticle package on \href{http://www.ctan.org/tex-archive/macros/latex/contrib/elsarticle}{CTAN}.}

%% Group authors per affiliation:
%\author{James T. Brown\fnref{myfootnote}}
%\author{John J. Beck\fnref{myfootnote}}
%\address{1600/1700 SW 23rd Dr., Gainesville, FL. 32608}
%\fntext[myfootnote]{Since 1880.}

%% or include affiliations in footnotes:
\author[usda,uf]{James T. Brown\corref{corresponding}}
\cortext[corresponding]{Corresponding author}
\ead{james.t.brown@usda.gov}

\author[uf]{Daniel A. Hahn}
\author[usda]{John J. Beck}

\address[usda]{USDA‐ARS Center for Medical, Agricultural, and Veterinary Entomology, 1700 SW 23rd Drive, Gainesville, FL 32608, USA}

\address[uf]{Department of Entomology and Nematology, University of Florida, Gainesville, FL 32611}



\begin{abstract}
European corn borers, \textit{Ostrinia nubilalis}, populations in the United States exist at almost every latitude east of the Rocky Mountains. Diapause is genetically determined and many insects use it to protect themselves from winter stress and take advantage of warm spring and summer seasons but the length and intensity of winter they experience varies by latitude. Different strains of European corn borers with different diapause genotypes have adapted to regional differences in winter timing through differences in diapause length. Before winter begins these pests prepare for diapause by preferentially storing lipids, however, the composition of lipids between diapause genotypes has yet to be determined. Our study compares European corn borer larvae with different diapause genotypes to determine whether diapause genotype influences the composition of stored lipids during diapause. If the timing and intensity of winter varies between the regions where these insects are found, then the composition of lipids stored should differ between genotypes. 
\end{abstract}

\begin{keyword}
\textttInsect European corn borer\sep overwintering\sep lipid storage\sep LC-ELSD\sep triglyceride\sep GC-FID\sep fatty acid methyl ester 
\MSC[2019] 00-01\sep  99-00
\end{keyword}

\end{frontmatter}

\linenumbers



\section{Introduction}
Temperate dwelling insects have adapted to avoid winter stress through diapause. Before winter begins, insects increase the storage of proteins, carbohydrates, and lipids before the onset of winter. European corn borer populations have been recorded at latitudes as far north as New York state and as far south as north Florida (Caffrey and Worthley 1927).  At the northernmost population distribution, winters are historically longer, colder, and less variable compared to the winters at lower latitudes across its population distribution (Hut et al. 2013; Levy et al. 2013). This part of its range is dominated by European corn borers with the long diapause genotype (Beck and Apple 1961). Larvae with the long-diapause genotype enter diapause earlier in the fall and exit diapause later in the spring compared the short-diapause genotype found at more southern latitudes (Levy et al. 2015). Diapause initiation for the short-diapause genotype requires a shorter photoperiod and lower temperatures than the long-diapause genotype. Previous work has revealed that European corn borers preparing for diapause store more lipids than non-diapause larvae. Among diapausing larvae, those with the long-diapause genotype store more lipid than larvae with the short diapause genotype however, differences in lipids composition between long and short-diapause genotypes has yet to be characterized (Brown et al. 2019 in preparation; Vukašinović et al. 2018).
Climate change is predicted to increased seasonal variability and temperatures  generate warmer temperatures. For insects warmer more variable temperatures could result in higher metabolic rates possibly driving up the amount of energy needed to fuel diapause preparation activities and diapause (Williams et al. 2012). 

<More to be added later….>

\section{Materials and Methods}
\subsection{Insect Collection and Rearing}
Larvae, pupae, and adult Ostrinia nubilalis specimens were collected from New York State courtesy of Dr. Erik Dopman's lab at Tufts University prior to 2015. Strain identity was determined by the Dopman lab using the pgFAR autosomal gene (Lassance et al. 2010; Wadsworth et al. 2015). For the duration of the experiment, colonies of the two genotypes were reared separately at 26 °C under a 16 hour light period and an 8 hour dark period (16:8 L:D photoperiod) to promote continuous development. 
Specimens used in the experiment  were collected as eggs from each colony and organized into biological cohorts. A biological cohort was defined as clutches of eggs oviposited on a single day by females of the same strain. Eggs from each biological cohort were held at 16L:8D photoperiod, 23 °C, and 65% relative humidity (rH) until they hatched. Upon hatching, each biological cohort was subdivided into two groups and reared in either the diapause-inducing treatment (12L:12D photoperiod, 23 °C, and 65% rH) or the non-diapause treatment (16L:8D photoperiod, 23 °C, and 65% rH). Within each treatment, larvae were reared together in groups and provided artificial European corn borer diet ad libitum (Frontier Agricultural Sciences, Newark, DE, Product F9478B). Individual larvae that reached the end of the fourth instar were transferred to a 32-well bioassay tray (Frontier Agricultural Sciences, Newark DE, Product RT32W), provisioned with more artificial diet and returned to their respective treatment conditions until sampling.

<More to be added later….>


Lipids were sampled from larvae in the non-diapause treatment were sampled at the start and end of the the fifth instar. In the diapause inducing treatment, larvae were sampled at the start of the fifth instar and at the start of dipause. were sam, at the end of the fifth instar  the onset of diapause, because energy stores are at their peak at the start of diapause. I diagnosed the onset of diapause in final larval instar larvae by assaying for the termination of frass production, which signifies the start of the wandering stage. The wandering stage is a developmental step that occurs at the end of the larval feeding stage in continuous developing larvae and those programmed for diapause \citep{Sakurai1998}. Larvae were removed from artificial diet and held in isolation for thirty minutes. After thirty minutes of isolation, larvae that did not produce frass were recorded as wandering. Using this wandering assay, I tracked the population of larvae for up to forty days and recorded the following developmental events: 1) the day that larvae eclose into the final instar, 2) the wandering day, and 3) pupation. All larval samples intended for lean mass and lipid measurements were assayed for wandering only once and larvae determined not to be wandering were removed from the experiment. 
\subsection{Lipid Extraction}
Sampled larvae were assigned a unique identifier and freeze-dried under vacuum to remove water. When the mass of each freeze-dried larvae varied by less than 1\% over a 24-hour period, the final dry mass measurement was recorded. After drying, 657 larval samples were then assigned to one of the 43 extraction cohorts and stored in a -80 $^\circ$C freezer. Each extraction cohort consisted of larvae from each biological cohort. Lean mass and lipid mass were measured for each larva sampled. First, lean mass was separated from lipid mass using a slightly modified Folch liquid-liquid extraction method \citep{FOLCH1957}. Larval samples were solubilized in pre-weighed microcentrifuge tubes (USA Scientific, Ocala, FL., 1420-8700) using a 3:1 solvent mixture of hexanes and methanol. The hexanes layer containing lipids was siphoned away from the methanol layer and collected in pre-weighed 15-mL glass vials (Millipore-Sigma, St. Louis, MO., 27347) and both layers were saved. Lean mass was estimated by drying away the methanol from the solubilized insect tissue and weighing the dry tissue powder. To estimate lipid mass, the hexanes were dried away from the lipids and the dry lipids were weighed. 

\subsection{Liquid Chromatography of Lipid Classes}
Step 1: Preparation
	Remove vials from SpeedVac, turn off SpeedVac properly
	Gather materials:
o	2 sets of LC vials, crimp caps, and crimper
	1 - labeled with sample ID
	2 - labeled sequentially with corresponding rep ID (Rep-#)
o	3 mL syringes and filters
o	Pasteur pipettes
o	Automatic pipette and 5 mL pipette tubes
o	Small beaker: REMOVE FROM OVEN
o	DCM
	Weigh and record extractions vials + dry
Step 2: Dilution
	Estimate volume needed to complete all the dilution at 4mL per vial. Plus, an additional 10 mL into beaker
	Unscrew caps on tubes: up to 6 vials at a time
	Add EXACTLY 4 mL of DCM to each vial
	Prepare syringe and filter; place into the intended storage vial
	Prepare Pasteur pipet
	Vortex extraction vial; 6 seconds 
o	Use this time to match the extraction vial ID to the LC vial ID
	Transfer 3mL into syringe and fill LC vial to 1.5mL
	Quickly move syringe and filter into LC run vial and fill with remaining dilution
	Crimp and seal vials TIGHTLY
Step 3: Clean Up
	Syringes and filters  biohazard waste
	Automatic pipette tips and Pastuer pipettes  glass disposal
	Extraction vials  into Liquinox
	Black septa caps  into Liquinox

\subsection{Gas Chromatography of Fatty Acids}
STEP 1: Preparation
•	Fill the water bath and turn on hotplate (73 = metal cup, 93 = square metal bowl)
•	Turn on SpeedVac refrigerator 
•	Add 400ul of sample into a reaction vial
•	Dry reaction vials on using SpeedVac for 15mins
•	Gather and label vials and test tubes

STEP 2: Derivitization
•	Add 400uL of KOH into each reaction vial
•	Warm each reaction vial for 15min at 55oC
•	Prepare test tubes with sodium sulfate
•	Cool  reaction vial in ice bath for 3min
•	SLOWLY add 200uL of H2SO4 into each reaction vial 
•	Cool reaction vial in ice bath after the addition of acid
•	Vortex each reaction vial for approx. 10sec to thoroughly neutralize reaction.

STEP 3: Storage
•	Add 3mL hexanes to each reaction vial
•	Vortex each reaction vial until mixed thoroughly
•	Allow reaction vial to rest for at least 5mins
•	Use a Pasteur pipette to draw off hexanes layer
•	Transfer hexanes layer into a test tubes containing sodium sulfate
•	Vortex hexanes and sodium sulfate layer to “dry” solution
•	Transfer “dry” hexanes into GC storage vials
o	Wrap vials in Parafilm and store in residential freezer
o	Prepare a GC storage vial of hexanes.

\subsection{Statistical Analysis}

\section{Results}
\subsection{Lipid classes extracted from short and long-diapause larvae}

\subsection{Fatty acids extracted from short and long-diapause larvae}

\section{Discussion}
\begin{enumerate}
    \item Answer: Lipid classes extracted from short and long-diapause larvae
    \item Answer: Fatty acids extracted from short and long-diapause larvae
    \item Previous work has has revealed increases in global lipid storage for diapausing ECB and larvae with the long-diapause genotype and lipid class differences among diapausing ECB has been characterized however, differences in lipid class composition between long diapause and short-diapause ECB has yet to be characterized.
    \item Triglyceride composition differences between diapause genotypes could offer insights about the role of stored triglycerides in ECB and other diapausing pest in adapting to variable winter stress.
    \item ***Warmer temperatures will increase metabolic activity during diapause preparations and during diapause. ***Warmer temperatures could provide ECB populations with warm summer days and food resources that last longer into the fall.
\end{enumerate}
\section{Bibliography styles}

There are various bibliography styles available. You can select the style of your choice in the preamble of this document. These styles are Elsevier styles based on standard styles like Harvard and Vancouver. Please use Bib\TeX\ to generate your bibliography and include DOIs whenever available.

Here are two sample references: \cite{Feynman1963118,Dirac1953888}.

\section*{References}

\bibliography{mybibfile}

\end{document}