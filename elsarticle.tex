\documentclass[review]{elsarticle}

\usepackage{lineno,hyperref}
\modulolinenumbers[5]

\journal{Comparative Biochemistry and Physiology Part B: Comparative Biochemistry}

%%%%%%%%%%%%%%%%%%%%%%%
%% Elsevier bibliography styles
%%%%%%%%%%%%%%%%%%%%%%%
%% To change the style, put a % in front of the second line of the current style and
%% remove the % from the second line of the style you would like to use.
%%%%%%%%%%%%%%%%%%%%%%%

%% Numbered
%\bibliographystyle{model1-num-names}

%% Numbered without titles
%\bibliographystyle{model1a-num-names}

%% Harvard
%\bibliographystyle{model2-names.bst}\biboptions{authoryear}

%% Vancouver numbered
%\usepackage{numcompress}\bibliographystyle{model3-num-names}

%% Vancouver name/year
%\usepackage{numcompress}\bibliographystyle{model4-names}\biboptions{authoryear}

%% APA style
%\bibliographystyle{model5-names}\biboptions{authoryear}

%% AMA style
%\usepackage{numcompress}\bibliographystyle{model6-num-names}

%% `Elsevier LaTeX' style
\bibliographystyle{elsarticle-num}
%%%%%%%%%%%%%%%%%%%%%%%

\begin{document}

\begin{frontmatter}

\title{Diapause genotype alters triglyceride accumulation during diapause in larvae of the European corn borer \textit{Ostrinia nubilalis} (Lepidoptera: Crambidae)}
\tnotetext[mytitlenote]{Fully documented templates are available in the elsarticle package on \href{http://www.ctan.org/tex-archive/macros/latex/contrib/elsarticle}{CTAN}.}

%% Group authors per affiliation:
%\author{James T. Brown\fnref{myfootnote}}
%\author{John J. Beck\fnref{myfootnote}}
%\address{1600/1700 SW 23rd Dr., Gainesville, FL. 32608}
%\fntext[myfootnote]{Since 1880.}

%% or include affiliations in footnotes:
\author[usda,uf]{James T. Brown\corref{corresponding}}
\cortext[corresponding]{Corresponding author}
\ead{james.t.brown@usda.gov}

\author[uf]{Daniel A. Hahn}
\author[usda]{John J. Beck}

\address[usda]{USDA‐ARS Center for Medical, Agricultural and Veterinary Entomology, 1700 SW 23rd Drive, Gainesville, FL 32608, USA}

\address[uf]{Department of Entomology and Nematology, University of Florida, Gainesville, FL 32611}

\begin{abstract}
The duration of winter varies regionally by latitude and many insects in these regions use diapause to protect themselves from winter stress while synchronizing their growth with seasons that support their growth and development. European corn borers, \textit{Ostrinia nubilalis}, occupy multiple latitudes and they have adapted to regionally specific winter lengths with a strain-specific diapause phenology that varies by diapause genotype. Our study compares diapause and non-diapause European corn borer larvae with genetically different diapause lengths to determine the extent to which diapause genotype and diapause initiation influences the composition and accumulation of lipid classes. We found that diapause was correlated with increased accumulation of total triglycerides, increased () fatty acids, and decreases in () fatty acids. Triglycerides (##) were the largest lipid class accumulated by diapausing larvae and non-diapausing, regardless  of diapause genotype. 
\end{abstract}

\begin{keyword}
\textttInsect \textit{Ostrinia nubilalis}\sep diapause\sep diapause genotype\sep 
Lipid classes\sep
\MSC[2019] 00-01\sep  99-00
\end{keyword}

\end{frontmatter}

\linenumbers

\section{Introduction}

\section{Materials and Methods}
\subsection{Insect Collection and Rearing}
\textit{Ostrinia nubilalis} eggs were provided as a courtesy from Dr. Erik Dopman's laboratory at Tufts University. The two genetically distinct European corn borer strains used during my experiment were collected as a mixture of larvae, pupae, and adults from New York State prior to 2015 and kept as separate colonies \citep{Wadsworth2015}. Strain identity was determined genotypically using the pgFAR autosomal gene \citep{Lassance2010}. This gene codes for an important enzyme involved in determining the female sex-pheromone blend and is partly responsible for maintaining strain differences. The pgFAR-Z allele is carried by the Z-strain larvae and the pgFAR-E allele is carried by the E-strain larvae, and each allele produces a distinct pheromone blend \citep{Lassance2010}. For the duration of the experiment, colonies of each genotype were reared at 26 $^\circ$C  under a 16L:8D photoperiod to promote continuous development. 

Individuals intended for experimentation were collected as eggs from each colony and organized into "biological cohorts". A biological cohort was defined as clutches of eggs oviposited on a single day by females of the same strain. Initially, eggs from each biological cohort were held under a 16L:8D photoperiod, 23 $^\circ$C and 65\% rH until they hatched. Upon hatching, each biological cohort was divided and reared in either the diapause-inducing treatment (12L:12D photoperiod, 23 $^\circ$C and 65\% rH) or the non-diapause treatment (16L:8D photoperiod, 23 $^\circ$C and 65\% rH). Larvae from each biological cohort were reared together in groups and provided artificial European corn borer diet ad libitum (Frontier Agricultural Sciences, Newark, DE, Product F9478B). When larvae from each biological cohort within each treatment reached the end of the fourth instar, they were separated and reared individually in 32-well bioassay trays (Frontier Agricultural Sciences, Newark DE, Product RT32W). Each well of the bioassay tray was provisioned with diet and returned to either diapause-inducing or non-diapause treatment conditions until sampling.

Stored energy was measured at the onset of diapause, because energy stores are at their peak at the start of diapause. I diagnosed the onset of diapause in final larval instar larvae by assaying for the termination of frass production, which signifies the start of the wandering stage. The wandering stage is a developmental step that occurs at the end of the larval feeding stage in continuous developing larvae and those programmed for diapause \citep{Sakurai1998}. Larvae were removed from artificial diet and held in isolation for thirty minutes. After thirty minutes of isolation, larvae that did not produce frass were recorded as wandering. Using this wandering assay, I tracked the population of larvae for up to forty days and recorded the following developmental events: 1) the day that larvae eclose into the final instar, 2) the wandering day, and 3) pupation. All larval samples intended for lean mass and lipid measurements were assayed for wandering only once and larvae determined not to be wandering were removed from the experiment. 

\subsection{Lipid Extraction}
Sampled larvae were assigned a unique identifier and freeze-dried under vacuum to remove water. When the mass of each freeze-dried larvae varied by less than 1\% over a 24-hour period, the final dry mass measurement was recorded. After drying, 657 larval samples were then assigned to one of the 43 extraction cohorts and stored in a -80 $^\circ$C freezer. Each extraction cohort consisted of larvae from each biological cohort. Lean mass and lipid mass were measured for each larva sampled. First, lean mass was separated from lipid mass using a slightly modified Folch liquid-liquid extraction method \citep{FOLCH1957}. Larval samples were solubilized in pre-weighed microcentrifuge tubes (USA Scientific, Ocala, FL., 1420-8700) using a 3:1 solvent mixture of hexanes and methanol. The hexanes layer containing lipids was siphoned away from the methanol layer and collected in pre-weighed 15-mL glass vials (Millipore-Sigma, St. Louis, MO., 27347) and both layers were saved. Lean mass was estimated by drying away the methanol from the solubilized insect tissue and weighing the dry tissue powder. To estimate lipid mass, the hexanes were dried away from the lipids and the dry lipids were weighed. 

\subsection{Liquid Chromatography of Lipid Classes}
Step 1: Preparation
	Remove vials from SpeedVac, turn off Speedvac properly
	Gather materials:
o	2 sets of LC vials, crimp caps, and crimper
	1 - labeled with sample ID
	2 - labeled sequentially with corresponding rep ID (Rep-#)
o	3 mL syringes and filters
o	Pasteur pipettes
o	Automatic pipette and 5 mL pipette tubes
o	Small beaker: REMOVE FROM OVEN
o	DCM
	Weigh and record extractions vials + dry
Step 2: Dilution
	Estimate volume needed to complete all the dilution at 4mL per vial. Plus, an additional 10 mL into beaker
	Unscrew caps on tubes: up to 6 vials at a time
	Add EXACTLY 4 mL of DCM to each vial
	Prepare syringe and filter; place into the intended storage vial
	Prepare Pasteur pipet
	Vortex extraction vial; 6 seconds 
o	Use this time to match the extraction vial ID to the LC vial ID
	Transfer 3mL into syringe and fill LC vial to 1.5mL
	Quickly move syringe and filter into LC run vial and fill with remaining dilution
	Crimp and seal vials TIGHTLY
Step 3: Clean Up
	Syringes and filters  biohazard waste
	Automatic pipette tips and Pastuer pipettes  glass disposal
	Extraction vials  into Liquinox
	Black septa caps  into Liquinox

\subsection{Gas Chromatography of Fatty Acids}
STEP 1: Preparation
•	Fill the water bath and turn on hotplate (73 = metal cup, 93 = square metal bowl)
•	Turn on SpeedVac refrigerator 
•	Add 400ul of sample into a reaction vial
•	Dry reaction vials on using SpeedVac for 15mins
•	Gather and label vials and test tubes

STEP 2: Derivitization
•	Add 400uL of KOH into each reaction vial
•	Warm each reaction vial for 15min at 55oC
•	Prepare test tubes with sodium sulfate
•	Cool  reaction vial in ice bath for 3min
•	SLOWLY add 200uL of H2SO4 into each reaction vial 
•	Cool reaction vial in ice bath after the addition of acid
•	Vortex each reaction vial for approx. 10sec to thoroughly neutralize reaction.

STEP 3: Storage
•	Add 3mL hexanes to each reaction vial
•	Vortex each reaction vial until mixed thoroughly
•	Allow reaction vial to rest for at least 5mins
•	Use a Pasteur pipette to draw off hexanes layer
•	Transfer hexanes layer into a test tubes containing sodium sulfate
•	Vortex hexanes and sodium sulfate layer to “dry” solution
•	Transfer “dry” hexanes into GC storage vials
o	Wrap vials in Parafilm and store in residential freezer
o	Prepare a GC storage vial of hexanes.

\subsection{Statistical Analysis}

\section{Results}
\subsection{Lipid classes extracted from short and long-diapause larvae}

\subsection{Fatty acids extracted from short and long-diapause larvae}

\section{Discussion}

\section{Bibliography styles}

There are various bibliography styles available. You can select the style of your choice in the preamble of this document. These styles are Elsevier styles based on standard styles like Harvard and Vancouver. Please use Bib\TeX\ to generate your bibliography and include DOIs whenever available.

Here are two sample references: \cite{Feynman1963118,Dirac1953888}.

\section*{References}

\bibliography{mybibfile}

\end{document}